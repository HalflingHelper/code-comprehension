  % APA 6th edition LaTeX template. Use this as a basis for writing documents that have to meet guidelines for the American Psychological Association. This has been optimized for "man" option, or manuscript format, which outputs a double-spaced, single column document that is designed for coursework or journal submissions. The inline comments assume you have some minimal working knowledge of LaTeX. Further, this is currently set up for managing references with biblatex and biber.

  % To be clear, this template is not endorsed by the American Psychological Association, the maintainers of the apa6 document class for LaTeX, or anyone else of great importance. For more documentation on using apa6, refer to its page at CTAN <http://www.ctan.org/tex-archive/macros/latex/contrib/apa6>.
  
  % ==============================================================
  %   Copyright (C) 2015  Jacob Long

  %   This program is free software: you can redistribute it and/or modify
  %   it under the terms of the GNU General Public License as published by
  %   the Free Software Foundation, either version 3 of the License, or
  %   (at your option) any later version.

  %   This program is distributed in the hope that it will be useful,
  %   but WITHOUT ANY WARRANTY; without even the implied warranty of
  %   MERCHANTABILITY or FITNESS FOR A PARTICULAR PURPOSE.  See the
  %   GNU General Public License for more details.

  %   You should have received a copy of the GNU General Public License
  %   along with this program.  If not, see <http://www.gnu.org/licenses/>.
  % ===============================================================


% "draftfirst" or "draftall" as option to watermark, 10pt/11pt/12pt for font size, noextraspace if there are spacing issues, "man" for regular papers (assignments, journal submissions), jou for journal-esque formatting.
\documentclass[man, 12pt, floatsintext]{apa7}

\usepackage[american]{babel}
\usepackage{amsmath}
\usepackage{csquotes}
\usepackage[style=apa,sortcites=true,sorting=nyt,backend=biber]{biblatex}

% Removes month from bibliography entries, which shouldn't be there for academic journals - optionally, remove the month entries from your .bib file on the offending files. Comment it out if using popular media, newspaper articles, etc. that need the month field. 
\AtEveryBibitem{
  \clearfield{month}
}
\DeclareLanguageMapping{american}{american-apa}

% Removes "retrieved from on date" from bibliography entry unless it is a wiki URL, which is closer to the spirit of APA's rule. See biblatex-apa documentation for more info.
\DeclareSourcemap{
\maps[datatype=bibtex]{
\map{
\step[fieldsource=url,
notmatch=\regexp{wiki},
final=1]
\step[fieldset=urldate, null]
}
}
}

% Add your BibTeX files here. Use source location if you aren't keeping them as the same folder as your document.
\addbibresource{bibliography.bib}

% Can help catch outdated code practices by giving you console warnings. Commented out by default so as to not confuse new users.
%\usepackage[l2tabu]{nag}

% title, etc.
\title{Understanding Loop Iteration Effects on Program Comprehension}
\shorttitle{Code Comprehension of Loops}
\author{Calvin Josenhans}

% The following four fields make up some of the front matter of your document. If working on an assignment for a course, I typically use "affiliation" for the class name. I have commented out abstract since minimal usage doesn't require it and leaving it blank will generate a blank page. Ignore the warning about the lack of abstract. 

\authorsaffiliations{Indiana University}
\abstract{Understanding how programmers comprehend code is important 
for improving software development efficiency and reducing cognitive load. 
In this study, we investigate how the number of loop iterations in a program 
affects code comprehension. 
We present a web-based experiment in which six experienced Python programmers were 
asked to predict the output of short programs involving iteration over lists. 
All programs were under 15 lines long and shared identical cyclomatic complexity, 
but varied in list length and in the use of additional state within the loop. 
Results show that increasing list length led to longer response times and lower accuracy, 
indicating a growing cost of mental simulation as the number of iterations increases. 
Programs requiring additional state tracking further reduced performance, 
suggesting increased working memory demands. 
We also observed significant learning effects across experimental blocks, 
with faster responses over time but no corresponding improvement in accuracy, 
due to increased speed of recognition of the program. }
\keywords{Code Comprehension, Cognitive Science}
%\authornote{}

\begin{document}
\maketitle

% \section{Introduction}
% No introduction heading in APA!

% What am I writing about
The development of computer programs is a significant and cognitively complex
undertaking. Programmers use various control-flow constructs, such as loops and conditionals, 
as well as data structures such as lists and trees to write programs that cover a broad range of 
disciplines and use cases. 

% Why is it important
Software development consists both of writing code, and of understanding the code that others have written.
It is not uncommon for many people to be working on the same parts of a code base and finding themselves
needing to comprehend code in order to make their own additions to it. Understanding this code comprehension
process and the factors that make it difficult allows for identification of areas where improvements
can be made to ease the cognitive load of comprehension, thus saving both time and money.

% What do I want the reader to know about it
We present a web-based experiment in which participants were asked to predict the output of several
short programs featuring iteration over lists. The programs were all under 15 lines long and had the
same cyclomatic complexity (\cite{mccabe_cyclo_1976}), however they differed in the number
of iterations of the loop that the subject had to simulate. We designed these programs
to answer the question of how the number of loop iterations affects code comprehension, and
to what extent that effect is dependent on additional factors of the program structure.


\subsection{Related Work}
% I don't need subsections within this, do I?

Research into programmers' mental models has resulted in the development
of various theories as to how programmers develop a mental model of a program
upon first observation, in order to be able to understand, debug, and modify it.
The bottom-up perspective posits that understanding starts with combining syntactic elements
of source code into higher and higher levels of abstraction of understanding. However other theories
propose that programmers have schema present for common program patterns and constructs, and
the activation of these informs the programer's model of the program (\cite{heinonen_lehtelä_hellas_fagerholm_2023}).

% Complexity studies and models
Studies have shown that there are so-called "rules of discourse" (\cite{soloway_ehrlich_1984}) 
that describe conventions that programming experts have that effect their understanding of programs.
Some of these are relevant for loop comprehension, such as the expectation that a looping 
construct such as \texttt{while} will only be used in cases where the body should be executed more than
once.
Research has also shown that in some cases expertise makes programmers more susceptible to 
certain kinds of errors where they have not been trained (\cite{hansen_goldstone_lumsdaine_2013}),
and that whitespace is relevant for judging whether statements are judged as being chunked together 
as part of the same loop body.

% We're curious about this loop thing
Research has also been devoted to understanding code complexity metrics, and how well
they correspond to or predict human comprehension. Models of mental simulation process
are developed, based on human subject performance at simulating containing loops with a limited number
of iterations (\cite{nakamura_monden_itoh_matsumoto_kanzaki_satoh_2003}), where in this case the goal was
to avoid effects of "loop induction" by the participants. This found a significant mental cost
incurred by non-constant variables.
An fMRI study comparing various code complexity metrics
measured subject brain activity while they simulated a program, some of which
contained loops (\cite{peitek_apel_parnin_brechmann_siegmund_2021}). 
Here similarly program vocabulary size was seen to burden working memory.


\section{Method}
\subsection{Subjects}
The participants in this study were six students at Indiana university, all of whom
self reported as having at least three years of experience with the Python programming language.
Additionally, all self-described as being either advanced or intermediate Python programmers,
with three falling into each category.

\subsection{Apparatus}
The studied was executed through the use of an online experiment created with 
the jsPsych framework\footnote{https://www.jspsych.org/latest/}. Participants completed 
the experiment on their own computers at a time of their own discretion.

\subsection{Design}
Two independent variables were manipulated within each subject.
\textbf{List length} was either 1, 5, 10, or 20, and describes the length of list 
in the code sample that was being iterated over.
\textbf{Program type} describes the type of task being carried out by the iteration in the program.
Three categories of programs were determined: \texttt{nostate} programs consisted counting of occurrences of 
simple arithmetic properties in a list; \texttt{count\_state} programs used an additional variable
to count certain pairings of numbers in a list, such as a number occuring twice in a row.
\texttt{hard} programs used both an additional state variable, and used the variable of 
loop operation for comparisons within the loop. All programs shared the same
cyclomatic complexity.
All code fragments used in the experiment can be found in our GitHub repository. \footnote{https://github.com/HalflingHelper/code-comprehension}

For each trial, the dependent variables measured were the response time, and whether or not
the given response was correct.

\subsection{Procedure}
Three blocks of conditions were created by combining base programs with
randomly generated lists of a particular length. Within each block no programs
were repeated, but each program occurred once per block, and would appear
in a random order. The program was presented on the screen with instructions to 
predict its output, which the subject could type in a text box,
and submit their final response. One second of wait time was
given between each program, and subjects could pause for as long as they
wanted between each of the three blocks.
% TODO: I don't know if there is much more to elaborate.


\section{Results}
Data analysis was performed in R. Any regressions were done with the built in 
\texttt{lm} function, and ANOVAs were performed with the \texttt{ez} package's
\texttt{ezANOVA} function.

\subsection{Accuracy}
%                Effect DFn DFd           F           p p<.05         ges
% 1           code_type   2  10  7.50000000 0.010240000     * 0.230986385
% 2               block   1   5  0.04424779 0.841696703       0.001561968
% 3                 len   1   5 37.80021686 0.001654732     * 0.365470992
% 4     code_type:block   2  10  4.19354839 0.047581069     * 0.139932059
% 5       code_type:len   2  10  2.79608837 0.108509123       0.113120286
% 6           block:len   1   5  0.14423077 0.719706949       0.001113982
% 7 code_type:block:len   2  10 14.46004320 0.001119776     * 0.199283136


%          Effect DFn DFd         F           p p<.05       ges
% 1     code_type   2  10  7.500000 0.010240000     * 0.3731814
% 2           len   1   5 37.800217 0.001654732     * 0.5330670
% 3 code_type:len   2  10  2.796088 0.108509123       0.2017971

Subject accuracy was measured as the percentage of correct responses to the programming problems.
Out of 216 total problems answered by our subjects, 66 of them, or 30.6\%, were given as incorrect answers.
We note, however, that 77.3\% of these were so-called "off-by-one" errors, as seen in figure \ref{fig:error_dist}.

\begin{figure}
    \centering
    \begin{minipage}{.45\textwidth}
           \caption{Distribution of errors}
    \includegraphics[width=2.5in]{images/error\_dist.png}
    % \figurenote{This is a great figure.}
    \label{fig:error_dist}
    \end{minipage}%
    \begin{minipage}{.45\textwidth}
        % I don't like this but whatever
    \caption{Accuracy effects}
    \includegraphics[width=2.5in]{images/accuracy\_by\_type.png}
    % \figurenote{This is a great figure.}
    \label{fig:accuracy}
    \end{minipage}
    \label{fig:interactions}
\end{figure}

We found a main effect of both code type ($p = 0.01$) and list length ($p=0.002$) on accuracy.
Accuracy decreased for code fragments with longer lists, and accuracy was lower for 
the \texttt{count\_state} and \texttt{hard} program types.
Though a statistically significant interaction between code type and list length was not found ($p=.11$), we do 
see a greater difference between the programs for longer lists.



\subsection{Response Time}
For analysis of response times we only considered correct responses to the program prompts.

% Main effects of code type and length within each block, no significant interaction
We found main effects of program type ($p=0.045$) and list length ($\text{intercept}=16324, \text{coef}=1125, p<0.05$) 
on response time. 
The \texttt{nostate} programs had significantly quicker response times than either of the 
other two program types, and as expected longer lists had longer response times, as there was more 
data for the subjects to process.

As with accuracy, we did not find a significant interaction between program type and list length ($p=0.42$).


\begin{figure}
    \centering
    \begin{minipage}{.45\textwidth}
          \caption{Program type and Response time}
    \includegraphics[width=2.5in]{images/prog\_type\_rt.png}
    % \figurenote{This is a great figure.}
    \label{fig:prog_type_rt}
    \end{minipage}%
    \begin{minipage}{.45\textwidth}
        % I don't like this but whatever
 \caption{List length and Response time}
    \includegraphics[width=2.5in]{images/code\_len\_rt.png}
    % \figurenote{This is a great figure.}
    \label{fig:code_len_rt}
    \end{minipage}
    % \label{fig:interactions}
\end{figure}


\subsection{Learning Effects}

We found a significant effect of block index on reaction time
($\text{intercept}=35802$, $\text{coef}=-10338$, $p<0.05$),
% Didn't improve, just got faster
indicating that subjects improved at the task over the course of the observation.
However there was no correlation between block index on subject accuracy, so while
speed increased, it did not necessarily result in improved performance.
 
For regressions of response time on list length, 
the intercepts stayed consistent across the three blocks, while the slope
decreased.


\begin{figure}
    \caption{Slopes and Intercepts}
    \includegraphics[width=2.5in]{images/fits.png}
    % \figurenote{This is a great figure.}
    \label{fig:fits}
\end{figure}


\section{Discussion}
Our results show that increasing the number of iterations in a program increases
the time taken for mental simulation of the program, indicating that to at least some extent,
the programmer is taking time to manually progress through loop iterations.

We observed significant learning effects on the tasks, and see that improvements in response time are
due to improved recognition of the particular program, rather than proceeding faster through each
iteration of the loop. This indicates that the subjects are perhaps building schema of the particular
programs we used here, even over the short course of the experiment.

With regards to accuracy at the task, we see incidents of incorrect answer increase with list length,
and no correlation between response time and accuracy. This suggests that mistakes are not a result
of subjects spending more time on the code as a whole, but due to errors in the operations they are simulating,
and even as that simulation speed increases, there is no corresponding improvement in accuracy. 
The error rates were higher as well in programs with more variables being used,
so as more working memory is occupied, errors are more likely to happen.

We were unable to determine whether the relationship between list length and response time was a strictly
linear one, or whether more complex behavior was at play, due to both limited participants, and a limited
selection of list lengths observed.


\subsection{Future Work}
The results here show promise for probing deeper into how programmers understand loop execution.
Further study might incorporate eye-tracking or fMRI measurements as other code comprehension
investigations have to draw physiological grounding for our observations here. There
is also a much greater variety of program structure available to probe, and perhaps
loops in the context of larger programs are understood differently. Additionally, though
subjects here were all fairly experienced programmers, investigation into whether the learning
effects we observed remain present in novices would also offer insight into how programmers
learn to adapt.

% This is where the bibliography goes. Finish the body of your paper before this point, other than the appendix. 
\printbibliography


% I have commented out the appendix section since it isn't a standard for minimal documents. 
%\appendix 

\end{document}

  % ==============================================================
  %   Copyright (C) 2015  Jacob Long

  %   This program is free software: you can redistribute it and/or modify
  %   it under the terms of the GNU General Public License as published by
  %   the Free Software Foundation, either version 3 of the License, or
  %   (at your option) any later version.

  %   This program is distributed in the hope that it will be useful,
  %   but WITHOUT ANY WARRANTY; without even the implied warranty of
  %   MERCHANTABILITY or FITNESS FOR A PARTICULAR PURPOSE.  See the
  %   GNU General Public License for more details.

  %   You should have received a copy of the GNU General Public License
  %   along with this program.  If not, see <http://www.gnu.org/licenses/>.
  % ===============================================================